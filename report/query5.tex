\section{Query 5}

Το πέμπτο ερώτημα είναι πιο σύνθετο καθώς συνδιάζει τρία σύνολα δεδομένων, το Crimes το "Census Blocks" και το "Median Household Income by Zip Code", για την εξαγωγή κοινωνικοοικονομικών συμπερασμάτων. Στόχος είναι η εύρεση της συσχέτισης μεταξύ του μέσου κατά κεφαλήν εισοδήματος με την ετήσια μέση αναλογία εγκλημάτων ανά άτομο σε κάθε περιοχή του Los Angeles. Για την υλοποίηση του ερωτήματος απαιτείται χρήση της βιβλιοθήκης Apache Sedona για τη δημιουργία Spatial Joins, ώστε να αντιστοιχιστεί κάθε έγκλημα στην περιοχή που ανήκει. Η ανάλυση του Physical Plan αποδεικνύει ότι ο optimizer λειτούργησε βέλτιστα επιλέγοντας στρατηγικές Broadcast για την ελαχιστοποίηση της κίνησης δεδομένων στο δίκτυο, αλλά και τη δημιουργία ευρετηρίου R-Tree στα δεδομένα αναφοράς για την γεωχωρική αναζήτηση. Καθοριστικό ρόλο στην απόδοση έπαιξε ο μηχανισμός Adaptive Query Execution. Κατά τη διάρκεια της εκτέλεσης, το AQE ανίχνευσε ότι το μέγεθος των δεδομένων προς ένωση ήταν μικρότερο από το αναμενόμενο. Έτσι, παρενέβη στο φυσικό πλάνο και αντικατέστησε αυτόματα το προγραμματισμένο SortMergeJoin με ένα Broadcast Join, εξαλείφοντας την ανάγκη για περιττό σάρωμα και επιταχύνοντας το ερώτημα.

Η εκτέλεση του ερωτήματος πραγματοποιήθηκε με 3 διαφορετικά configurations, διατηρώντας σταθερούς τους συνολικούς πόρους και αλλάζοντας την κατανομή. Παίρνουμε τους χρόνους που φαίνονται στον πίνακα \ref{query5_table}.

\begin{table}[H]
	\centering
	\caption{Χρόνοι εκτέλεσης των διαφορετικών configuration}
	\begin{tabular}{|c|c|}
		\hline
		Configuration & Time (s)\\
		\hline
		2 executors, 4 core,  8GB mem & 49.22\\
		4 executors, 2 cores, 4GB mem & 54.48\\
		8 executors, 1 cores, 2GB mem & 55.67\\
		\hline
	\end{tabular}
	\label{query5_table}
\end{table}

Παρατηρούμε ότι το πρώτο configuration έχει καλύτερο χρόνο εκτέλεσης, ενώ τα άλλα δύο έχουν σχεδόν ίδιο. Παρόλο που το άθροισμα των υπολογιστικώων πόρων παραμένει σταθερό σε όλα τα σενάρια, η υπεροχή του πρώτου configuration οφείλεται στη βέλτιστη διαχείριση των Broadcast Joins. Tα δεδομένα ανφοράς και το χωρικό ευρετήριο μεταφέρθηκαν και αποθηκεύτηκαν στην μνήμη μόνο 2 φορές, εξυπηρετώντας αποδιτικά τα threads του κάθε executor. Αντίθετα στις άλλες δύο περιπτώσεις απαιτείται μεταφορά και δέσμευση μνήμης για τις ίδιες δομές 4 και 8 φορές αντίστοιχα, αυξάνοντας το network overhead. 